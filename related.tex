\chapter{Antecedentes} 
\label{cap:related}
%\todo{
% Hay varias cosas en relación a la estructura de este punto que debes mejorar.
% Te planteo primero los problemas y luego una posible solcuicón:
% 1. No entiendo porque hablas de las la RA y de la radiolaogia diagnostica en le punto 1.1. 
% 2. No entiendo porque el punto 1.1.1 y el 1.3 están separados. 

%Solución 
%1.1 Simuladores Médicos de Realidad Virtual
%1.1.1 Técnicas tradicionales (sin hablar de simuladores RV, lo que tengas pasaló al actual 1.1.3 )
%1.1.2 Simuladores RV (Aquí va todo lo de G. Burdea)
%1.1.3 Simuladores médicos (fusión de los actuales 1.1.1 y 1.3) cuando los fusiones que no te quede muy repetitivo. Vuelvelo a leer.
%1.1.3.1 Aprendizaje (no me gusta mucho el título)
%1.1.3.1 Evaluación. Mete validez dentro
%1.2 Caracterización de los procedimientos simulados
%1.2.1 RA
%1.2.3 x-ray (los titulos antiguos)
%1.3 RASimAs
%1.4 Como está
%}


En este capítulo se repasará la bibliografía con el objetivo del ayudar al lector a entender mejor el alcance de esta tesis.
Además, se explicará en detalle la estructura del proyecto RASimAs, haciendo especial hincapié en las partes desarrolladas en este trabajo. El lector también podrá encontrar una revisión bibliográfica en la que se repasan los principales avances que se han producido en los últimos años en el área en la que se enmarca esta tesis, destacando sus fortalezas y debilidades. 

Se va a comenzar dando una perspectiva general de los simuladores de realidad virtual para luego centrarse en simuladores particulares de especial relevancia para el proyecto presentado. A continuación, se introducirá brevemente conceptos relacionados con el aprendizaje en este tipo de plataformas. También se detallará la evaluación a los que se someten los simuladores de realidad virtual, en concreto en el ámbito médico para probar su efectividad. Posteriormente se introducirá al en lector los procedimientos médicos en los que se ha trabajado en esta tesis. Además, se resumirá la división de tareas del proyecto \ac{RASimAs}. 
Por último, se introducirá al lector en el estado del arte  en la animación de personajes dentro de la informática gráfica, haciendo hincapié en aquellas técnicas relacionadas con el método propuesto en este proyecto. Esta sección se centrará en describir las etapas de la animación esqueletal en las que está basado el algoritmo propuesto en esta tesis y las distintas propuestas que existen para automatizarlas y, por último, se compararán con los métodos de animación basados en modelos físicos.


