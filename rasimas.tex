\subsubsection{RASIMAS}
\label{art:rasimas}

\todo{Aunque aquí hagas una pequeña descripción, es importante que describas la RA y la Radiología diagnostica en el estado del arte. Deben ir los pasos detallados del procedimiento. Pon alguna cita, aquí y en el estado del arte. Por ejemplo, el CV de RA de Cork. Esto luego lo usaras en el courseware}

The utility of ultrasound-guided RA as an alternative to electrical nerve stimulation has
increased in popularity over the past 5 years. The use of ultrasound imaging for nerve
localization is an innovative application of an old technology that addresses some of the
shortcomings of the traditional electrical nerve stimulation techniques. The single most
important advantage of ultrasound for nerve blocks is the ability to confirm local anaesthetic
spread around the target nerve. Besides imaging the needle and nerve, ultrasound clearly
reveals the surrounding hazardous structures, including blood vessels, pleura, and viscera

Performing RA follows a full procedure that may be unknown to non anaesthesiologists. In
order to give an overview of this procedure, the outlines of an US guided single-shot FNB
has been characterised in the following. The objective of this characterisation is to inform non 

anaesthesiologists of the project as to the required functionality of a procedural skills trainer
relevant to peripheral nerve block. This description commences once the patient and
equipment has been appropriately positioned and prepared, vital signs monitoring attached
and intravenous access established. The description finishes once the local anaesthetic has
been injected. There are important components of ultrasound guided femoral nerve block
which occur before and following the needling procedure itself. These may indeed need to be
included in a truly valid description of the procedure, however for the purposes of this
characterisation, the description will focus on the above start and end points (Fig. 1).


In the case of US guided training, on entering the training function, Dr ABC is prompted to
interact with the haptic devices within the VR environment. Dr ABC is then guided through
the key-enabling skills relevant to ultrasound-guided femoral nerve block as follows:
1. Scout scan
2. Optimisation of obtained image (PART manoeuvre)
3. Identification of anatomical features of interest
4. Immobilisation of probe and selection of needle insertion site
5. Insertion of needle and advancement of needle-in-plane toward the target, while needle
tip and shaft are visualised
6. Halting needle advancement at the ‘appropriate’ location
7. Injecting local anaesthetic and correctly identifying the pattern of spread as satisfactory or
unsatisfactory.

\subsubsection{Objetivos del simulador}
Simulator: general principles
The user needs for both procedures detailed above allow us to extract the general principles
for the simulator. This section provides therefore general guidelines for the design of the
simulator. In a process to anticipate the step of the design of the simulator, a list of possible
features is already proposed in a second step.
6.1 Description
6.1.1 Objective
1. To enable novice anaesthetists acquire and develop the necessary procedural skills to
perform proficiently a simulated ultrasound-guided or nerve stimulation-based local nerve
block, prior to performing regional anaesthesia in the clinical arena (starting with the
femoral nerve block, additional blocks and/or nerve localisation techniques to follow on
subsequent iterations).
2. To permit anaesthetists who have been out of the clinical practice of ultrasound-guided or
nerve stimulation-based nerve block for some time to relearn the procedural skills
necessary to perform this technique and to define the point that proficient performance
has been attained.
3. To assess performance metrics associated with proficient performance of ultrasoundguided
and nerve stimulation-based nerve blocks.
4. To allow regional anaesthesia programme directors monitor trainee performance over
time
6.1.2 Location of use
We intend providing the virtual reality simulator for regional anaesthesia to a large audience.
To allow a broad spectrum of anaesthetists using the simulator as training tool, the simulator
need to be relatively inexpensive, facilitate uncomplicated use, robust and reasonable small
with limited need of maintenance. Specifically, the simulator should be primarily used in:
1. Academic teaching hospitals
2. Colleges of Anaesthesia (or equivalent professional training bodies)
3. Academic institutions investigating procedural skills training
6.1.3 Principles of use
1. RASim must be contextually relevant and function within the existing or evolving regional
anaesthesia curriculum
2. RASim training must be based upon appropriately detailed characterised procedures
3. RASim training must allow the development of relevant procedural skills and associated
clinical decision making
4. RASim training must never allow the development of procedural skill relevant only within
the simulation environmentRASim training should allow performance related feedback, repeated practice and
performance tracking and progression over time each based on precisely defined metrics
6. RASim based training programmes must have a defined beginning and end (entry and
exit) constituting proficiency based progression for defined applications e.g. independent
practice , supervised practice
6.2 Features
Anticipating the next step consisting in designing the future VR platform, a non-exhaustive
list of possible features for the simulator have been extracted based on the user needs, for
which technological choices will have to be made. The features are grouped in tables by
themes as follows:
General features
Working space
Virtual patient
Visualisation features
Electric Nerve Stimulation features
US guidance features
Haptic feedback features
Assessment \& Education features
Portability features
US guidance procedure
Electrical stimulation guidance procedure
Software/Hardware features

% Logic
% This section analyses the logic of the Guided Mode. The Courseware is implemented
% as a finite state machine and we detail the events that are tracked in order to know,
% during the simulation, which is the current stage of the procedure to gather the
% appropriate metrics and to display appropriate supportive information. In this section,
% we explain the events associated with: the assessment metrics (GM.AM.XX), the
% error messages (GM.ER.XX) and the information displayed on user demand
% (GM.IUD.XX). Additional information on the error metrics is detailed in Section 4.1.2
% and further information on error messages and on information displayed on users
% demand can be found in Section 4.1.3.
% With the aim of facilitating the read of this section the states of the system are divided
% in three phases: Scan-scout, needle guidance and injection.
% Scan-scout:
% GM.Scan.1. Welcoming and Theatre Preparation. The patient and the US
% machine have to be positioned appropriately in the room. The 

% application shows a video explaining this concept. The users can
% replay the video as many times as they need. The stage ends when
% the users close the window where the video is displayed. This stage
% neither records any metrics, nor provides additional feedback.
% GM.Scan.2. Ultrasound Machine Configuration. The users select the
% appropriate US machine configuration values and turn the US
% machine on. Then the users click on the “Continue” button. The
% courseware does not let the user jump into the next step until
% appropriate values are set.
% i. Error messages: GM.ER.01.
% ii. Information on user demand: GM.IUD.01.
% iii. Assessment metrics: GM.AM.01, GM.AM.02, GM.AM.42.
% GM.Scan.3. Aseptic technique. The application shows a video explaining this
% concept. The users can replay the video as many times as they
% need. The stage ends when the users close the window where the
% video is displayed. This stage neither records any metrics, nor
% provides additional feedback.
% GM.Scan.4. Transducer above the inguinal crest. The users are asked to
% place the ultrasound probe over the inguinal crest. The simulation
% jumps into the next step if the users place the US transducer over
% the iliac crest for 1,5 seconds.
% i. Error messages: GM.ER.02.
% ii. Information on user demand: GM.IUD.02, GM.IUD.03.
% iii. Assessment metrics: GM.AM.03, GM.AM.04, GM.AM.42.
% GM.Scan.5. Placing the transducer on patient skin. The users are asked to
% place the US probe on the patient skin. In order to have a cross
% sectional view of the relevant anatomy, they must check the US
% rotation, alignment and tilt. The system jumps into the next step
% when the probe is touching the correct anatomical area for 1,5
% seconds.
% i. Error messages: GM.ER.02, GM.ER.03, GM.ER.04, GM.ER.05.
% ii. Information on user demand: GM.IUD.02, GM.IUD.03.
% iii. Assessment metrics: GM.AM.03, GM.AM.04, GM.AM.05,
% GM.AM.06, GM.AM.07, GM.AM.42.
% GM.Scan.6. Sidedness Check. The users check if the probe sidedness is
% correct moving the transducer laterally on the patient skin. The
% users click on the “Continue” button after adjusting the probe
% sidedness. The system does not allow stepping into the next stage
% until the users adjust correct probe sidedness.
% i. Error messages: GM.ER.02, GM.ER.03, GM.ER.04, GM.ER.05,
% GM.ER.06.