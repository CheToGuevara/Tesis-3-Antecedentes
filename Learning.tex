\section{Entrenamiento y evaluación}
\label{art:learning}


En anteriores secciones se ha revisado el uso de simuladores en medicina aunque es bien sabido que el uso en otro tipo de contextos no es nuevo. Desde el primer simulador de vuelo que aparece en 1929 en la industria de la aviación \cite{page2000brief}, el principal objetivo de estos simuladores es garantizar la seguridad e intentar mejorar el aprendizaje y la respuesta antes errores poco comunes.

En \cite{donaldson2000err} se cifra en cientos de miles las muertes ocurridas en hospitales estadounidenses como consecuencia de errores médicos, eso sin contar posibles otro tipo de daños a los pacientes que implican gastos económicos. En este libro se planteaba la necesidad de mejorar la formación de los profesionales para evitar este tipo de errores. 
Es necesario también garantizar la seguridad y la intimidad de los pacientes durante el proceso de aprendizaje lo cual lleva implícito la exigencia ética que representa el propio código deontológico del personal sanitario. 

En el libro \cite{dent2017practical} se presenta una serie de objetivos que debería cumplir una institución de enseñanza médica al formular el currículum del personal sanitario.
\begin{itemize}
    \item Producir nuevos profesional sanitarios.
    \item Utilizar los procedimientos médicos modernos
    \item Cumplir la regulación gubernamental
    \item Asegurar que los estudiantes puedan completar el curso
    \item Satisfacerlas expectativas de los pacientes
\end{itemize}

Estos objetivos se enfrentan con una serie de problemas que han impulsado a las instituciones médicas a potenciar el uso de los simuladores.
Por una parte,  desde hace décadas se vienen disminuyendo el tiempo de trabajo para los profesionales en formación, que a la vez reduce el tiempo con pacientes reales. Además, como podría pensarse ha habido cambios respecto a que un paciente sea objetivo de exploraciones y procedimientos redundantes con objeto de entrenar a estudiantes, siendo una molestia y constituye un peligro. Por otra parte, ciertas organizaciones han cambiado sus métodos de evaluaciones con acreditaciones y certificaciones frente a la clásica evaluación basada en el conocimiento exclusivamente.  También hay que añadir que movimientos por los derechos de los animales han hecho restringir el uso de estos con motivos de aprendizaje.

Otro motivo por el cual es necesario cambiar las formas de aprendizaje de la medicina es porque esta también ha evolucionado con el tiempo.
Tradicionalmente, el plan de estudios de la medicina era incremental desde los aspecto más básicos hasta llegar a la especialización. Aunque esto antes era lo más aceptado, avances recientes en esta disciplina han incrementado notablemente el contenido del currículum. Esto ha hecho que aparecieran escuelas para cada especialidad médica donde el conocimiento médico ha crecido exponencialmente en las últimas décadas y los métodos tradicionales de enseñanza no son completamente adecuados para manejar esa cantidad de material.

Obligados a buscar alternativas para garantizar una exposición clínica variada y completa, junto con el desarrollo de la investigación en el campo de la simulación, se están presentando nuevos modelos de simulación cada vez mejores y más realistas.

Los simuladores se han mostrado como métodos adicionales de enseñanza frente a los métodos tradicionales, esto hace que nuevas formas de aprendizaje requieren nuevas formas de validación de estos métodos. Dentro de este campo, se han desarrollado ampliamente instrumentos los cuales pueden probar y validar de forma objetiva la diversidad de competencias y objetivos que tienen que cumplir.


\subsection{Aprendizaje}

Frente a un modelo asistencial en la formación de los estudiantes, los simuladores presentan  ventajas educativas que convierten estas herramientas en ideales para enfrentarse a los problemas anteriormente citados. Estas herramientas han demostrado que pueden reducir el tiempo de aprendizaje\cite{} siendo posible además la repetición del ejercicio tantas veces como sea necesario sin necesidad de esperar intervenir con pacientes reales. Además, la curva de aprendizaje son menores que las curvas presentadas en el entrenamiento clásico.

Por otra parte, los simuladores permiten que el estudiante cometa errores sin las consecuencias reales que podrían resultar en un entorno con pacientes. El alumno se puede enfrentar a todo tipo de situaciones, desde las introductorias a las más complicadas donde errar no es crítico. Citando a \cite{ziv2008educacion} "Los errores son experiencias de aprendizaje y ofrecen grandes oportunidades de mejorar a través del aprendizaje de los mismos". 

Además, este método permite que el alumno reciba valoraciones y comentarios en tiempo real de sus profesores y pueda reflexionar en ese momento. Adicionalmente, los simuladores presentan un entorno educativo estandarizado, reproducible y objetivo que permite una evaluación del estudiante constante.

Según \cite{pales2010uso}hay una serie de condiciones que van son necesarios para que los simuladores sean un método eficaz de enseñanza.

\begin{itemize}
\item Los simuladores se tienen que basar en una planificación estricta junto con los objetivos docentes. Deben tener un guión que refleje claramente la situación a simular, los objetivos marcados y las competencias que se van a entrenar.

\item Las habilidades entrenadas tienen que estar integradas en el currículum  y se debe planificar la enseñanza de diferentes habilidades complementariamente a la enseñanza teórica. Lo que se enseña debe ser relevante en el contexto

\item La evaluación es una parte esencial del proceso como en cualquier otra actividad educativa. La retroalimentación es una de las partes imprescindibles de la simulación
\item Cualquier simulador no puede estar aislados del entorno clínico real y se debe ser consciente de sus limitaciones y de las limitaciones de la tecnología. El manejarse correctamente con el simulador no es igual a competencia clínica.
\item Los usuarios deben ser conscientes de que aunque trabajan en un entorno de simulación, han de actuar de la misma manera como lo harían en la realidad. El material de simulación no puede considerarse como un mero juguete y en su manejo han de observarse las mismas condiciones de uso y seguridad que en la realidad.

\end{itemize}

%Sofia says:
Deliberate Practice and the Acquisition and Maintenance of Expert Performance in Medicine and Related Domains
Ericsson, K Anders

Deliberative practice «Ericsson»( practicar muchas horas no es suficiente, hay que practicar fijándose en las cosas a reforzar y en lo que hay que mejorar... y se requieren un número mínimo de horas para llegar a convertirse en experto, estas horas siempre va ser más barato entrenarlas con un simulador, una vez adquirido y descontado el coste inicial, por lo que un simulador es interesante cuando mucha gente la practicar muchas horas).

Además es más seguro para el paciente (patient safety).

ZPD zone of proximal development «vugotsky» ( cuando queremos aprender algo, necesitamos que sus conocimientos caigan dentro de lo que se llama la zona de desarrollo próximo, porque si es demasiado fácil lo que vamos a aprender no nos motiva, y si es demasiado difícil, nos descorazonamos). 

Por eso se deben presentar los andamios (Scaffolding, «jerome bruner») necesarios para poder construir nuevo conocimiento y que dicho conocimiento caiga dentro de esa zona,  que es como la zona a nuestro alcance.

Transferability of skills ( que lo aprendido como simulador, sea extrapolable a la situación real, en nuestro caso, a un quirófano)

Decay of skills (para cuando ya saben y son expertos, pero al estar un tiempo sin practicar, para que puedan practicar con el simulador y así ponerse al día en las habilidades y que no decaigan).


A pesar de todas estas posibilidades, algunas de ellas muy complejas, debemos tener en cuenta que muchas veces no se siempre se requieren modelos demasiados complejos para el entrenamiento de determinadas habilidades. La simulación es una metodología docente, el simulador, sea de la complejidad que sea, un mero instrumento. Para cada objetivo docente hay un modelo de simulador suficiente y apropiado. El mérito de un simulador no es su complejidad sino su utilidad y la frecuencia y aceptación para su uso por parte de los profesores. No es recomendable basar toda la enseñanza en el uso de simuladores específicos.


\todo{comentamos diferencias entre evaluación formativa y sumativa} 

\todo{reescribir esta frase para introducir el siguiente punto}Esto nos lleva al siguiente punto, donde se expone los métodos de evaluación a los que se someten estos simuladores para ser aptos.

\subsection{Evaluación}

Para la evaluación de una herramienta (puede ser un test, un simulador o cualquier herramienta que sirva para evaluar) resulta necesario saber que es lo qué se quiere evaluar. Se quiere reflejar que en el caso del simulador, \todo{reescribir} realice lo que se requiera que los estudiantes necesiten entrenar o aprender. Además, la diversidad que presentan los simuladores también significa que se necesitarían diferentes métodos de evaluación. Por tanto, se tiene que adecuar las evaluaciones para cada caso concreto y asegurar que la evaluación sea válida, fiable y factible. 
Cuantos más criterios de evaluación se realice al nuevo método de enseñanza que se presenta, más seguro se estará de que los resultados son un fiel reflejo del correcto desempeño de los estudiantes.

Muchos autores toman como referencia de una buena evaluación el consenso establecido en  \cite{norcini2011criteria}, en la que se siguen los siguientes criterios para una buena evaluación.

\begin{itemize}
\item Validez: El simulador es válido para la tarea a la que ha sido diseñado.
\item Reproducibilidad o consistencia: El simulador presenta la misma evaluación en circunstancias similares.
\item Equivalencia: Presenta la misma clasificación o puntuación a través de diferentes instituciones o análisis.
\item Viabilidad: El simulador es práctico, realista y razonable dada las circunstancias y el contexto.
\item Efecto educativo: El simulador motive aquellos usuarios consiguiendo un beneficio educativo.
\item Efecto \emph{Catalytic}: El simulador aporta resultados y retroalimentaciones que crea, mejora y respalda la formación, e impulsa el aprendizaje.
\item Aceptación: Stakeholders \todo{como traduzco esto?} encuentran el simulador y los resultados creíbles.
\end{itemize}

Muchos de estos criterios han sido descritos previamente, pero en \cite{norcini2011criteria}, se quiere hacer hincapié en el \emph{efecto catalítico}, ya que aquel instrumento de evaluación que consigue mejorar el aprendizaje además de hacer que el usuario se sienta cómodo con el instrumento es algo beneficioso para el entrenamiento.



\subsubsection{Validez}\todo{terminar}

Los simuladores se pueden tratar como herramientas de evaluación como puede ser un test, y al igual que estos necesitan ser evaluados para validar que realizan la tarea para los que fueron diseñados. Estas validaciones tienen que ser constantes en el tiempo y en el lugar 
Estas no son simples ya que como se puede leer en \cite{pales2010uso}  hay multitud de tipos de validación citan como las más importantes:
\begin{itemize}
    \item Validez aparente
    Esta validación se basa en la apariencia. Si los usuarios, tanto estudiantes como profesores, sienten que una herramienta o test es similar, razonable y relevante, esto motivará el entorno de enseñanza y aprendizaje. Esta validación está determinada por la respuesta dada por todos aquellos usuarios que están involucrados con la herramienta.
    
    \item Validez de contenido. La herramienta presenta parámetros totalmente relacionados con el contenido que se quiere medir.
    
    Esta validación mide si el alcance de la herramienta es 
    Un simulador 
    
    
    % Content validity
% This refers to the extent to which a test or examination actually
% measures the intended content area. For an examination to have
% content validity it must have 'item validity' and 'sampling
% validity'. These terms are best explained in the following
% example. If a test is designed to measure knowledge of the
% human anatomy then good item validity is present if all the
% quesfions deal with facts
% pertaining to the human body. However, poor sampling validity
% will be apparent if all the questions focus on the lower limbs.
% The issue of content should be seen in relation to:
% · the subject matter you are teaching
% · the students who are being taught.
% How to establish content validity
% · Define the subject matter being assessed.
% · Identify the cognitive /behavioural/attitudinal process
% involved.
% · Establish the outcomes expected
    \item Validez predictiva.
    Con esta validación se pretende demostrar que una herramienta  es capaz de predecir el desempeño de los usuarios. Para lo cual, se realiza primero una sesión con la herramienta nueva y se predice el resultado que mostrará con otra herramienta ya establecida.
    
    % Predictive validity relates to the certainty with which a test can
% predict future performance. It is particularly important if you
% are using your assessment for selection purposes. N9 test will
% have perfect predictability so it is wise to base any decision on
% more than one predictor. The procedure is to:
% · Administer your test.
% · Collect measures of the new behaviour.
% · Correlate the two results.
% · The magnitude of the correlation coefficient will determine
% the predictive validity.

\item Validez concurrente.
    Esta validación pretende comparar el desempeño de la herramienta nueva por evaluar con una ya establecida. Se procede a que el mismo conjunto de usuarios realicen las actividades de las dos herramientas, comparándose entonces los resultados obtenidos esperando una correlación entre ambos. Tanto la validez predictiva como la validez concurrente se consideran validez de criterio.
    
    % Concurrent validity
% This refers to the degree to which scores on a test correlate
% with the scores on an established test administered at the same
% time. The procedure is to:
% · Administer the new test.
% · Administer the established test.
% · Correlate the two sets of scores.
% · The greater the positive correlation, the greater the validity

    \item Validez de constructo.
    Consiste en que la herramienta cumpla con la hipótesis planteada dónde se espera que la herramienta sea capaz de medir aquello por el cual está diseñada. 
    
% Se refiere al grado en que el instrumento de medida
%cumple con las hipótesis que cabría esperar para un %instrumento de medida diseñado
%para medir precisamente aquello que deseaba medir. 
% Construct validity
% Construct validity is the extent to which a test measures a
% hypothetical construct (e.g. empathy, intelligence) or a trait that
% explains behaviour but is not easily observed. For example, if a
% theory of schizophrenia hypothesised that high scorers on a test
% will take longer to problem-solve than low scorers, then if high
% scorers do indeed take longer it would provide evidence for
% construct validity. 




%It can be difficult to determine and all forms
% of validity should be used as evidence for its presence.

    
    
\end{itemize}



% Whatever assessment instrument is used it must be valid for the
% task it is to do. In other words the answer to the question 'Am I
% measuring what I am supposed to be measuring?' must be
% positive. A particular examination might be valid for one
% purpose but invalid for another. For example, a series of MCQs
% which test factual recall may.i)e a valid measure of whether a 
% student has read a textbook on diabetes but invalid as the
% indicator of whether that same student can actually manage a
% patient suffering from diabetes.
% The measure of validity is not a straightforward process as a
% variety of types of validity are described and 'degrees of
% validity' are recognised. Building up a dossier to support a
% claim of validity can involve looking at five major types:
% · content validity
% · concurrent validity
% · predictive validity
% · construct validity
% · face validity.
% It is important to be familiar with all of these but the emphasis
% attributed to each is dependent on the reasons for assessing






